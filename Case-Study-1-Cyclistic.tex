% Options for packages loaded elsewhere
\PassOptionsToPackage{unicode}{hyperref}
\PassOptionsToPackage{hyphens}{url}
%
\documentclass[
]{article}
\usepackage{amsmath,amssymb}
\usepackage{iftex}
\ifPDFTeX
  \usepackage[T1]{fontenc}
  \usepackage[utf8]{inputenc}
  \usepackage{textcomp} % provide euro and other symbols
\else % if luatex or xetex
  \usepackage{unicode-math} % this also loads fontspec
  \defaultfontfeatures{Scale=MatchLowercase}
  \defaultfontfeatures[\rmfamily]{Ligatures=TeX,Scale=1}
\fi
\usepackage{lmodern}
\ifPDFTeX\else
  % xetex/luatex font selection
\fi
% Use upquote if available, for straight quotes in verbatim environments
\IfFileExists{upquote.sty}{\usepackage{upquote}}{}
\IfFileExists{microtype.sty}{% use microtype if available
  \usepackage[]{microtype}
  \UseMicrotypeSet[protrusion]{basicmath} % disable protrusion for tt fonts
}{}
\makeatletter
\@ifundefined{KOMAClassName}{% if non-KOMA class
  \IfFileExists{parskip.sty}{%
    \usepackage{parskip}
  }{% else
    \setlength{\parindent}{0pt}
    \setlength{\parskip}{6pt plus 2pt minus 1pt}}
}{% if KOMA class
  \KOMAoptions{parskip=half}}
\makeatother
\usepackage{xcolor}
\usepackage[margin=1in]{geometry}
\usepackage{graphicx}
\makeatletter
\def\maxwidth{\ifdim\Gin@nat@width>\linewidth\linewidth\else\Gin@nat@width\fi}
\def\maxheight{\ifdim\Gin@nat@height>\textheight\textheight\else\Gin@nat@height\fi}
\makeatother
% Scale images if necessary, so that they will not overflow the page
% margins by default, and it is still possible to overwrite the defaults
% using explicit options in \includegraphics[width, height, ...]{}
\setkeys{Gin}{width=\maxwidth,height=\maxheight,keepaspectratio}
% Set default figure placement to htbp
\makeatletter
\def\fps@figure{htbp}
\makeatother
\setlength{\emergencystretch}{3em} % prevent overfull lines
\providecommand{\tightlist}{%
  \setlength{\itemsep}{0pt}\setlength{\parskip}{0pt}}
\setcounter{secnumdepth}{-\maxdimen} % remove section numbering
\ifLuaTeX
  \usepackage{selnolig}  % disable illegal ligatures
\fi
\IfFileExists{bookmark.sty}{\usepackage{bookmark}}{\usepackage{hyperref}}
\IfFileExists{xurl.sty}{\usepackage{xurl}}{} % add URL line breaks if available
\urlstyle{same}
\hypersetup{
  pdftitle={Case Study 1 - Cyclistic Marketing},
  pdfauthor={Salimah Ismail},
  hidelinks,
  pdfcreator={LaTeX via pandoc}}

\title{Case Study 1 - Cyclistic Marketing}
\author{Salimah Ismail}
\date{2023-10-12}

\begin{document}
\maketitle

\hypertarget{case-study-1-how-does-a-bike-share-navigate-speedy-success}{%
\section{Case Study 1: How Does a Bike-Share Navigate Speedy
Success?}\label{case-study-1-how-does-a-bike-share-navigate-speedy-success}}

Cyclistic is a fictional bike-share company in Chicago who has several
pricing plans: single ride passes, full-day passes and annual
memberships. Customers who purchase single rides or full-day passes are
considered casual members, whereas those who have annual memberships are
considered Cyclistic members.

\hypertarget{the-task}{%
\subsection{The Task}\label{the-task}}

Cyclistic believes that there is a growth opportunity in converting
casual riders into full Cyclistic members. This analysis aims to
determine \textbf{the main differences between how casual members use
Cyclistic versus annual members} so that the Cyclistic marketing
department can understand these differences and market annual
memberships to these casual customers effectively.

\hypertarget{the-data}{%
\subsection{The Data}\label{the-data}}

Cyclistic data sets for analysis are available
\href{https://divvy-tripdata.s3.amazonaws.com/index.html}{here}, mainly
organized as one file per month. I chose to examine one year of data,
from July 2022 - 2023 to get the most recent insights on riders.

\hypertarget{available-information}{%
\subsubsection{Available Information}\label{available-information}}

The data sets include information such as: - Electric vs.~Classic Bike -
Start and End times of the rides - Starting and Ending locations
(station, latitude, longitude) - Membership Type

\hypertarget{limitations}{%
\subsubsection{Limitations}\label{limitations}}

\begin{itemize}
\tightlist
\item
  We cannot attach the different trips to members beyond their
  membership type. Therefore, we cannot tell how frequently an
  individual rider is utilizing the service. It would have been great to
  be able to look at how frequently a casual member uses the service
  compared to an annual member.
\item
  There are many instances where station information is missing.
  However, starting latitudes and longitudes are available, so we can
  determine distance traveled based on those calculations.
\end{itemize}

\hypertarget{method}{%
\subsection{Method}\label{method}}

\textbf{1. The data sets were combined into one CSV file.}

\begin{verbatim}
## 
## Attaching package: 'dplyr'
\end{verbatim}

\begin{verbatim}
## The following objects are masked from 'package:stats':
## 
##     filter, lag
\end{verbatim}

\begin{verbatim}
## The following objects are masked from 'package:base':
## 
##     intersect, setdiff, setequal, union
\end{verbatim}

\begin{verbatim}
## Rows: 6547094 Columns: 13
## -- Column specification --------------------------------------------------------
## Delimiter: ","
## chr  (7): ride_id, rideable_type, start_station_name, start_station_id, end_...
## dbl  (4): start_lat, start_lng, end_lat, end_lng
## dttm (2): started_at, ended_at
## 
## i Use `spec()` to retrieve the full column specification for this data.
## i Specify the column types or set `show_col_types = FALSE` to quiet this message.
\end{verbatim}

\textbf{2. Some additional columns were added to the data frame
including:} - Ride length, by calculating the difference between the
start and end time - Day of the week the ride occurred - Month that the
ride occurred - Ride distance, by calculating the difference between the
starting and ending latitude and longitudes.

\textbf{3. Some rides that had errors in it were discarded, such as ones
with negative times travelled even though the distance was significant.*
}

\textbf{4. Summary statistics were calculated and relevant obeservations
grouped by Casual and Annual members were visualized such as:}

\emph{Ride Length Comparisons}

\begin{verbatim}
## Don't know how to automatically pick scale for object of type <difftime>.
## Defaulting to continuous.
\end{verbatim}

\includegraphics{Case-Study-1-Cyclistic_files/figure-latex/boxplot-ride-length-1.pdf}

\emph{Distance Comparisons}

\includegraphics{Case-Study-1-Cyclistic_files/figure-latex/boxplot-distance-length-1.pdf}

\emph{Bike Type Usage Comparisons:}

\includegraphics{Case-Study-1-Cyclistic_files/figure-latex/bar-chart-bike-type-1.pdf}

\emph{Bike Day Usage Comparisons:}

\includegraphics{Case-Study-1-Cyclistic_files/figure-latex/histogram-bike-day-1.pdf}

\hypertarget{conclusions}{%
\subsection{Conclusions}\label{conclusions}}

\begin{itemize}
\tightlist
\item
  Casual Members, on average, seem to use the bikes for \emph{longer
  periods of time} than Annual Members.
\item
  Casual Members, on average, seem to travel about the \emph{same}
  distance as Annual members.
\item
  Casual Members seem to \emph{prefer electric bikes} proportionally
  more than Annual Members.
\item
  Casual Members tend to use the service \emph{in high volumes on
  weekends} as compared to Annual Members whose higher usage occur on
  weekdays.
\end{itemize}

\end{document}
